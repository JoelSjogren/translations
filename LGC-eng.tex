\documentclass{article}
\usepackage[T1]{fontenc}
\usepackage[style=french]{csquotes}
\usepackage{amsmath}
\usepackage{amssymb}

\author{Jean-Yves Girard}
\title{Logic as a geometry of cognition \\ (La Logique comme Géométrie du Cognitif) \\ Manifest}
\date{22 March 2004}

\begin{document}

\maketitle
\noindent
(Unofficial translation.)\footnote{}

\section{The big cemeteries under the moon}
Replace the philosophy at the center of the scientific activity, restore the philosophy of science, as programme! For this, we propose to reactivate the major tool which constitutes \emph{logic}, by freeing it from the rut of the \enquote{linguistic turn}; this reactivation would be by means of geometry, a \enquote{geometric turn} of a sort. The geometrization becomes possible ---and above all necessary--- through the eruption of the \emph{cognitive}, which makes obsolete the old paradigm of verismo/dualism/atomism/essentialism inherited from the 20th century, the century of \emph{scientism}.

Scientism seeks answers, while \emph{science} would instead seek questions. Around 1900, one would seek the \enquote{ultimate solutions}\footnote{}, all questions must find their answers, a little like maidens their husbands. Scientism, which cannot make any mistakes, tackled racial issues in 1904: 80,000 Hereros hanging in Namibia, in the name of eugenics... and it was nothing like the third Reich. And what to say about the final solution to class inequality, the extermination of the \enquote{Koulaks}, in the name of Marxist \enquote{science}, in the early 1930s? The paradigm under our critique is not affiliated with any of the murderous \enquote{-isms} of the past century; but it is fed by the same breast, an unprecedented regression of thought, accompanied by technical progress, also unprecedented.

\section{An equally long absence}
Last century thus saw an accentuated gap between science and the reflection about science. On the one hand, the (often) excellent scientists who devote themselves to Jivaro reductionism, on the other the (always) cultivated philosphers who imagine themselves being able to talk about science based on data collected (at best) from popular writings. Poles of the same battery, if they are opposite... cheerfully, they agree on the essential: philosophy is not a very serious activity. Tweedledum thinks that it is very nice, but not vital: [???], William of Ockham! Tweedledee sees the scientist as a technician, a kind of cook who prepares the dishes, but whose taste is not good enough to appreciate them fairly.

To say that a scientist does not understand what he does, is a bit steep. The new theories don't all emerge in the same way by natural selection... The scientist \enquote{knows} what he does, even if he expresses it in \enquote{his own way}: thus, often, a scientist superego blurs his message. But it must not fall any more into the inverse excess and pretend that the communication is not a futile activity. Because finally, we have to communicate with colleagues, and especially \emph{with ourselves}; this communication is the methodological reflection which orients the work, puts it in abyss, it is the choice of problems, the supporting directions, and it is well part of \enquote{philosophy}. And, before a fragmentation of knowledge without precedent, the reflection on science becomes a priority, as well for the seasoned \enquote{scholar} as for the beginner who \enquote{enters into science}.

For sure, the time of great \enquote{honest men} is over, there will be no more of Descartes, of Pascal, of Leibniz. The specialization is such that when a scientist dominates the principal branches of his domain, without even making sparks, he passes for a Pico della Mirandola... Extended, but we must plead a sort of \enquote{cultural exception} about the philosophy of science. If there is no longer a philosopher-mathematician-physicist-astronomer, we can dream of a \enquote{mediation} between these irremediably decoupled activities.

\section{The linguistic turn}
The mediator between science and philosophy, this could be the logician. In fact, logic ---with a an etymology too much resembling that of language--- would it not define itself as the intersection between science and philosophy? Be that as it may, it is well placed for this ---and between us, it is not very original.

This remark underlies the \enquote{linguistic turn} of the past century, a moment of extreme importance. The linguistic turn, in giving to language a leading position, is opposed to \emph{verismo}\footnote{} in its manners, if not convincing, then at least sincere originally. Very quickly the project has found its limits: to reduce everything to properties of language, this is bold and interesting, but it confers on these properties a dubious extraterritoriality. One speaks of \enquote{meta} properties, and of meta-meta, in a manner of a Russian doll, each relieving the next, the turning point is set to... turn in circles, like a broken record, a foundational loop where two turtles are chasing each other and biting the other's tail, believing to advance while they are stuck in place. This is what we call Gödel's theorem of 1931: formally, it registers the failure of \enquote{extraterritoriality}. After this date, the \enquote{turning} is no more, at a conscious level at least, than a purely scholastic, academic activity.

The dominant vision ---counter-reform after Gödel's theorem--- conforms to Nestorian theology: everything rests on a trinity of Semantics/Syntax/Meta. The Son (or Word) reflects only imperfectly his father (Semantics), this is the incompleteness of the Son as not being consubstantial. Luckily the Holy Spirit is there, a little rogue, to muddy the waters: it is the Meta, the Polyfilla of cracks in the foundations. This vision has attracted cordial contempt from the scientists, who prefer to pass for overweight Platonists rather than fueling the meta.

The call on the meta returns to admitting the primitive side, resisting at every analysis, logical operations\footnote{}: on layers of verismo (the inevitable call on the theory of sets, seen \emph{infra}) and dualism (the opposition of syntax/semantics) it superimposes a doubtful \emph{essentialism}... After Nestorius, Thomas Aquinas.

\section{Augustine vs Thomas}
The remedy turns out to be worse than evil, there is something rotten in the longuistic turn. Looking closely, we may distinguish two aspects, one part of which being a \emph{proceduralism} which by nature is closer to existentialism, that of Augustin\footnote{}. The other part is an atomism which would like to reduce complex to simple, big to small: one thinks that a galaxy is made of stars, and not conversely; this atomism has failed in physics, in biology\footnote{}. It is responsible for the essentialist regression, for the prominence of the \enquote{meta} in the \emph{linguistic turn}.

Some logicians, and non the less, thinking about the Gödel/Bernays debate\footnote{}, have an Augustinian position: \enquote{the proof is prior to the thing being proved\footnote{}}, as opposed to the Thomism of Tarski or Kreisel\footnote{}. The Augustinianism in logic consists of a priority toward \emph{protocols} ---which are necessarily a little linguistic in nature--- but without giving them meaning \emph{a priori}: the meaning appears by polarization, when we focus on certain protocols, by \enquote{breaking the symmetry} so to speak. For an essentialist, the protocol, the formalism, the proof, follows the law, obeys it; the existentialist is more \enquote{western}\footnote{}: he hangs along and judges later, the things are what they are and not what they should be. Why is this trend so small? One must admit, because of the same atomism which has raged elsewhere: we can not give a revolutionary explanation of the world based on three pieces of string, by despising the contributions of real mathematics, of physics, and especially of the quantum. To present a consistent \enquote{Augustinian} point of view, it is necessary that existence preceeds essence, but which existence are we talking about? Surely not that of the 26 letters of the alphabet! Of the quest for geometry.

\section{The cognitive}
But while some are sleeping..., the evening stone turns in its [...]\footnote{}: but not in circles, insensitively, it puts itself to point in another direction, the \emph{cognitive}.

If verismo supposed a strict segregation between object and subject, we may define \emph{a contrario} the cognitive as the objectivization of the subject, which becomes an object as part of the whole. So, the human brain does not appear as more than an artifact of chemistry, realizing in an approximate fashion an ideal subject, it is the Subject himself, consubstantial with the object: the cognitive is not Nestorian. In the cognitive, we find ourselves among words like \enquote{I know}, \enquote{I believe}, with the conditional, with the distinction between perfect/imperfect... But it is always the physics of the 20th century which was cognitive. Already chaos (Poincaré) announced a \emph{practical} impossibility, that of predicting lottery drawings, or more nobly the long-term evolution of the solar system. Then relativity restrained (Einstein, 1905) derealized time and space, the mass. But, after all, Galilean mechanics was already a form of relativism, and, [...?] 21st century, well before 1917, Riemann had had the intuition of general relativity. As for the quantum (Heisenberg \& al, 1925), it no longer simply attacks the \emph{feasibility} of measurement, or absolute character of the result, it refuses until the object has been measured, it is the \emph{uncertainty principle}, which [...] More recently, computer science also has things mixed up, in effect, what is a computer, if not a subject-object?

\section{The (anti-)cognitive verismo}
The cognitive being the same negation as verismo, it is thus natural that we have searched to reconcile them, or rather to \emph{choke} the cognitive in a realist straitjacket.

A first example: we have found an analogy between cognitive phenomena and formal provability, this which is undeniable, but the comparison is not without reason. So is there an analogy between an iron a loudspeaker, both using an alternating current; yet a speaker plugged into 220V, this is not great. Here it's not about refusing the process ---consubstantial with science--- of reduction of a new phenomenon; it's simply about ---when this process only leads to atrocities--- admitting the irreducible novelty, and that's all. So, was it ligitimate to try and reduce the quantum to thermodynamics by means of \enquote{hidden variables}; but not to insist in the face of irrefragible failures.

valeurs de veritéThe know/prove analogy suggests a cognitive counterpart to models. Therefore we have attempted to reduce the cognitive to verismo by means of new truth values (\texttt{true, false, please wait, too late}), with modalities ---the condoms\footnote{} of logic--- (\texttt{for sure, this could be})... and, every time, it is not that. One arrives at best at laborious metaphores\footnote{} with every time an \emph{ad hoc} (il-)logical system, i.e., centered around the metaphor, which turns out to be a metaphor in itself\footnote{}. A single honorable failure, the \emph{Kripke models} ---an funny idea of 1950, very superior to epistemic \enquote{logic} and other indignities--- , but which shows itself to be sterile. One wants to talk about potentials, conditionals: it is simple, we make a list of all possible worlds. But, if the list of possibilities is already there, it makes one wonder where the potential is hidden, it is shy or what? In a problem which demands a radical Yalta redistribution of object/subject, we are content to just put mustaches on the objects. All the same, the problem with logical time and temporal \enquote{logic}; this grand mystery of nature, time, is evacuated bureaucratically, we index everything by time. As the song goes \enquote{Si l'on pouvait arrêter les aiguilles [if we could stop the needles]}... here, it is clocks that secrete time. The small cognitive vibration is recreated by external means, the models which obey the indices \emph{perinde ac cadaver}..., the models of Loyola in a way.

Quantum mechanics has been ---in fact always, hasn't it, Claude Allègre?--- an object of visceral refusal on the part of verists, because of a detail ---after all minor--- its non-determinism\footnote{}. In parallel with \enquote{hidden variables}, the logicians have tried to \enquote{drink the spell} by means of twistedvalues of verismo... Poor von Neumann, he was hardly inspired when he created quantum \enquote{logic}... it is true that he hardly persevered in this way, and that we also owe him algebraic eponyms, and there, it is tough. We know this which has happened in quantum logic, or rather we don't even know. The original sin was verist in nature: we keep the paradigm of verismo values, we will simply replace Boolean algebra by the orthogonal projectors of Hilbert space; this is a bit as if we had attached a fan to a wheelbarrow: we can't put anything more inside and it doesn't fly however. The good idea (the von Neumann algebras) is not far away, but it does not return in a dualiste frame of semantics/syntax. Besides, whatever the imperfections of the explanation called \enquote{of Copenhagen}, it is evident that the quantum can not accomodate a swing of object/subject.

\section{The computer science challenge}
More belatedly, computer science [fr: informatique] had to take the relays by launching a new cognitive challenge. Ass much as computer languages are very similar to formal languages, the programming does not treat itself as a verist, in true/false. Thus, that which interests the computer scientist is not the \emph{veracity} of information on the Internet, but rather its accessibility, its reproducibility... Imperceptibly, the boolean value as basis has passed from \texttt{true/false} to \texttt{left/right}, that is to say only the opposition between the two remains relevant, just like spin up/spin down as basis. Suddenly, a school of thought, going back to Poincaré, Brouwer,..., \emph{intuitionism}, lost since long in quarrels of the chapel, resurfaced. For the subjectivist, and thus inept, intuitionism became derealist, \emph{procedural}, Augustinian. What is procedurality? It is the fact that the machine interacts with... other machiens, by respecting protocols, and that nothing else matters. The rest, this which we see ---or rather \emph{believe} that we see--- is not relevant, all that matters is the dialogue of machines.

Let's take the example of a cognitivist failure: the computer scientists convinced themselves, when studying databases, that \enquote{missing information is false}. In effect, a bank is able to say \enquote{Mr. Kurz is not a client of ours} even though it does not have a file of all its non-clients. An essential cognitivist remark, immediately screwed up, by the equation of \enquote{knowing} with \enquote{proving}. The \enquote{non-monotone logics} resemble a remake of the Hilbert program ---which was based on a related idea, refuted by Gödels theorem---, but with the underlings [fr: troisième couteaux] of logic: if a property is not provable, then its negation is. Technically, the error is easy to understand, it \emph{suffices} to force the equivalence \texttt{true = provable}, which forces to pass over the body of undecidable properties, that correspond to \enquote{loops}, finite or infinite, of the calculus: we postulate therefore this sea serpent, the \enquote{loop detector}... But a diagonal argument ---going back to Cantor, Russell, Gödel--- produces for each \enquote{detector} a loop which escapes it: \emph{exit} non-monotone logics. But why then have we desired to force this equivalence between truth and provability? Common sense rebels against this: so a bank will not find a client by her maiden name, although she is recorded in the data concerning her; likewise, there is no absolute notion of \enquote{presence of a file on a hard disk}, the police may search it for \enquote{deleted} images, but which persist at least partially, while their owner beileves them destroyed. In other words, the \enquote{no} of the bank, the software of the search, refers to its internal procedures and nothing else. Wanting it to be otherwise is both torturing the logic and scorning \enquote{reality}. The bank does not say anything about \emph{the} quote, i.e. the response which it gives refers to a mode of exploration, to a search procedure.

The \emph{algorithmic complexity} is another example of a cognitive problem. After considerations of efficacy, we have been lead to classify the algorithms, for example, based on computation times, thereby algorithms in \emph{polynomial time}, and the whole world has heard about the famous \texttt{P = NP?} problem. The approaches to this question are verist, e.g., \enquote{finite models}; they haven't given much, we don't even have a \emph{handy} definition of polynomial time algorithms. The complexity has a strange status, that of a set theory (seen from the \emph{infra}) with a main attachment on the back\footnote{}: we don't have rights to the exponential function, it \enquote{costs} too much. But would there be an \emph{intrinsic} reason, mathematical, for refusing the exponential function and non-polynomial algorithms? This may be the hidden meaning of the question, without any real concrete stake, \texttt{P = NP?}. Of course, this reason would be of a forcibly cognitive nature.

It is possible that the challenge of computer science is as profound as that of the quantum. Besides, the emergence of a \emph{quantum calculus} ---very theoretical at the moment--- might aspire to a convergence of the principles behind the two activities\footnote{}.

\section{Set-theoretic atomism}
Superficially, the \enquote{linguistic turn} opposes itself against verismo, an idea moreover revolutionary for the epoch. But let's scratch a little, and we find a profound layer, very 21st century, very atomist, the theory of sets, like the government in Bordeaux of 1940: in \enquote{Nestorian theology}, the Father is a set theorist. The error of the linguistic turners, this is to not have ever delivered a cause for this moral pervasiveness of sets and biases since the beginning of the 20th century.

These biases were not, at the beginning, even foundational. If the focus of the \textbf{ZF} system is done essentially in 1908, by Zermelo, it is in the 21st century that it is developing, based on the works of Cantor on \enquote{exceptional} sets: do the values of a function determine their Fourier series, and in that case, may one ignore some? Nothing really foundational here. There was also a real need for clarification after the discovery of \enquote{free riders} in the worls of analysis, e.g., a function without derivative; this was the time of Bolzano, Weierstraß, Peano. To the end with \enquote{we can see that}: if a curve does not have a tangent, one does not \enquote{see} anything! A colossal work has been performed to define rigorously all mathematical notions ---thinking especially about Dedekind--- in an atomist spirit: define the the big terms of the small, the complex in terms of the simple.

The success of the theory of sets is undeniable, it is the \enquote{language of the foundational level} of mathematics: according to evidence, everything may be written in \textbf{ZF}. The theory of sets accounces a unity of mathematics\footnote{}, but only a \emph{unity in principle}. It represents more of a possibility than a reality: one does \emph{not} write, or only little, of mathematics in set theory; but one \emph{could}. This is to say that one may \emph{translate} into the theory of sets; but, \emph{traduttore traditore}, still bestowing \emph{ipso facto} a foundational role.

With the discovery of \enquote{free riders} in the second half of the 21st century, it became \emph{legitimate} to be self-critical about the validity of geometrical intuition, for instance about the pertinence of the notion of \emph{dimension}. We discovered that, from a strictly set theoretical point of view, the notion does not make sense (every infinite set is equipotent with its square). Furthermore, ther Peano \enquote{curve}, which \enquote{covers} a surface, broke the notion of dimension in the topological sense; but the massacre ends there, for the Peano curve does not induce a \emph{homeomorphism}\footnote{}, and algebraic topology had, in the 21st century, to show that the balls of different dimensions are not homeomorphic.

Things seemed settled, the dimension does not exist from the point of view of a set theorist, or from the point of view of measure, on the other hand it earned its meaning in the topological sense, and even more so, in the metrical sense. There is however something unsaid in all this: one admits that a mathematical object is a set (of points), on which one places a \enquote{structure}. It works, but is it correct? Say that a planar triangle is the set of its points, this is possible, and it is the (atomistic) choice of set theory; but we may just as well say that a point is the set of triangles that contain it, and besides, this holds up remarkably well: by passing to \emph{polars}, it is impossible to say whetehr a triangle is made of points or a point of triangles! Visibly, planar geometry does not speak about sets, and let's remember how, for the Greeks, a point was only the intersection of two triangles or the extremities of a line segment. Coming back to topology, it is evident that a sphere is not the \emph{set} $\{(x,y,z)\;;\;x^2+y^2+z^2=1\}$, however one may \emph{associate} this set to the sphere. Instead of seeing the sphere as a set to which one associates homology groups..., to which one associates a material representation, a set-theoretic \enquote{reification}. In other words, instead of declaring the anteriority of the set (the \enquote{semantics}) to the group (the \enquote{syntax}), we may turn the paradigm around. Is it the egg which made the chicken or the chicken which made the egg? We would tend to finally send them back to back, the set and its groups of invariants.

\section{Non-commutative geometry}
This observation is rendered a bit obsolete by \emph{non-commutative geometry}. The classical example is that of a \emph{torus} i.e. a mathematical air chamber; if we cut with scissors and follow a constant orientation, the result will depend on the angle of attack: if it is badly chosen (most common case) we never finish cutting the torus into a strip ever so thin; in other works we create a dense trajectory i.e. which seems to pass everywhere, whereas it is not a \enquote{Peano curve}. As if the torus was \enquote{too tight}, as if it ran out of points. But wemay not find the \enquote{missing points}, and it is the same idea of toroidal set which we have to question, to introduce non-set-theoretic, \enquote{non-commutative}, toruses says Connes. Technically speaking, a torus in the usual sense may be seen as the space of \enquote{smooth} functions, which is a commutative algebra. If we omit the commutativity, the algebras remain manipulable, but they don't come from a \enquote{real} variety like the torus, they are not \enquote{reifiable}.

This example should suffice to convince us that we are attesting a real expulsion of sets and the beginning of a new foundational approach, a harmony with the quantum miracle. As a matter of fact, Grothendieck in his time has already intended to expell sets to the benefit of \emph{categories}: sadly his \emph{topoi} are \enquote{reifiable}, i.e., they still have a \enquote{natural} set-theoretic substrate, that which we could not find for the algebras of operators.


But \emph{what} about the \emph{commutative}? Operators commute when they are \enquote{diagonalized} by a common \enquote{basis}\footnote{}. The unsaid common in logic, in the theory of sets, of categories, is the implicit agreement on such a \enquote{basis}. All these beautiful things, sets, elements, proofs, models, language, objects, morphisms, functions, arguments... \enquote{commute}. We don't agree on anything, except on this basic, set-theoretic \enquote{arena}, where everything is played out, everything is measured. Everyone is thus based on the same location, but let's imagine a chock that shifts the gyroscopes... the questions no longer fall exactly on their answers, the marbles in their boxes. However, if logic is also Augustinian like the physical world, the interaction takes place despite the absence of a formal status. In physics, we know that it happens by means of the \emph{reducion of the wave packet}. This is what to import into logic to spice up the object/subject relation!

\section{The geometric turn}
This expression should not be taken as a rejection of the linguistic turn, to which we owe the fundamental advances. It is more of a reform, after the fossilisation that we have talked about. The linguistic idea is excellent, it is the first step of any derealisation, but as any ideology it has its cellars, its non-spoken subliminals. And here the unspoken relates to meaning. Without truly admitting it, we have supposed that language came out of an empty world, stupidly egalitarian, all ideas becoming uniformly ugly when we have coded them sufficiently, bureaucratized them, this is the atomism that we have already denounced, and let's remomber that the 20st century was also the century of Brezhnev. So do the perfectly uneducated people pretend that mathematics is $2+2=4$, thereby showing a confusion between a calcuation and a theorem; the logicians, ---seemingly--- very sophisticated after the linguistic turn, are rather saying that $2+2=5$ because, isn't it, language is arbitrary... Recall these people choosing their notation, they write a tensor product as $\oplus$ and a direct sum as $\otimes$, there is no favouritism in the land of symbols!

That all is nothing but language, or at least, that in the manner of a table in a chinese restaurant, the world offers itself to us in various equivalent forms ---including language--- in a way that can be apprehended from various angles, here is a respectable thesis, which we may hold without much risk to the linguistic turn. In return, we will make the hypothesis that language is structured, that it is not that bureaucratic desert which we have just reached. But, what structure, where to search? Surely not in a verbose description of language, which leads to a pretentious and sterile essentialism. What remains is \emph{geometry}. By geometry, we don't intend anything very precise, let's say geometry is that which is sensitive to being coded, that which resists being coded.

Geometry, this is the discovery of simple structures, of symmetries. Let's give an example: anyone thinks a little about logic knows that the implication $\forall x \exists y \implies \exists y \forall x$ is incorrect. But why exactly? The current explanation is that in the first case the variable $y$ depends on $x$, and in the second case, it is independent of $x$... an absurd argument! [fr: et voilà pourquoi votre fille est muette!] However, one may say more about this sad paraphrase: the universal quantifier $\forall x$ is negative, passive: it waits for something, or in other words, \enquote{give me a value $a$ for $x$ and I will give you a value for $y$}; before acting, it is in the realm of the implicit, since it refers to an $x=a$ which is unknown. Nothing like $\exists y$, which is positive, active, explicit: $\exists y$ goes to say \enquote{I have a value $b$ for $y$, I am not showing it to you, it lies at the bottom of my suitcase, but if you insist you can have it}. Positive and negative progress differently, one may always postpone the positive (the making of the decision) or, if on prefers to advance the questions, a group $+-$ may thus be replaced by a $-+$, thus, $\exists y \forall x$ implies $\forall x \exists y$, but the converse is false. This has an immense mnemonic value, as it is difficult to remember the behavior of quantification in relation to modalities; from this we remark that $\Box$ is positive, since we know the direction is $\Box\forall\implies\forall\Box$, and not the inverse.

We will give an example among others of that which we call \enquote{geometry}: the logical propositions are divided into two classes, negative or positive, following their polarity.

\section{Object vs subject}
Finally, this is the connection between object/subject, the very place of this connection that we must question. Around 1900, it was reasonable to imagine mathematical objects studied by a subject who is proving theorems. This which has finally given way to a swing between syntax/semantics, organized around the properties of exchange of truth/provability: \emph{correction} \enquote{that which is provable is true} and \emph{completeness} \enquote{that which is true is provable}. \emph{Incompleteness} shows the limits of this vision, and introduces a third, the meta, which we invoke on days when the weather is bad, i.e., permanently. This is crippling, glass half-full or half-empty, matter of taste, in any case beneath the original pretentions of the \enquote{linguistic turn}. The quantum, the cognitive, give object status to the subject, not accidentally, but essentially. In the process of objectification of the subject, computer science has played a major role: we were brought back to revise the interpretative paradigms. Thereby, or by a choice between demonstration and counter-model, we preferred a dialectic model, in the very symmetric form of a play between reputedly homogeneous partners. But, the Copernician revolution has not been done, we use the expression \enquote{game semantics}, which suggests a sort of double trigger semantics: the players already express the opposition between syntax/semantics, if the game is a semantics it may not have a meta-semantics (in a meta language?)... Decidedly, there are some old clothes that stick to the skin, here again essentialism.

The \emph{procedural} interpretation of logic, i.e., the logic as a logic of rules, and not as its proper meta, makes profound structures apparent, such as the aforementioned polarities. In terms of \emph{ludics}, i.e., of games, polarity is the distinction between active/passive, \enquote{I play} vs \enquote{you play}. It is also the distinction \enquote{I produce} vs \enquote{I observe}, object/subject. This distinction is fundamental and at the same time very contingent, in effect, the logical dynamics supposes an exchange of roles (I play, then you play,...); this means that the same logical statement will, in the course of its interactions, pass without stop from the status of subject to that of object and \emph{vice-versa}, discovering, such a \enquote{logical onion}, more and more profound skins. This which seems to reopen the cellar where the bond between logic/quantum sleeps: in effect, if the basical logical operations correspond to a polarization of object/subject, we include better the irreducible status which the quantum allows to the observer. Immediately, we should give a more precise sense to these remarks, these analogies, and attempt a quantum interpretation of logic (and not a logical interpretation of the quantum). In particular, try to clarify the quantum status of logical polarization, in relation, maybe with the distinction of fermion/boson.

\section{The fundamental intuitions}
In the last chapter of \emph{Retour du divin} (1943), Audiberti develops the myth of submarine, a kind of metaphysical hell, from which one cannot escape, the most general dreams already have a place in the ship which \enquote{embarks all}.

Logic, the language paradigm, the theory of sets, the \enquote{commutative} world, this is a little like this submarine. We would like to leave, but, \emph{modulo} the translations of the 21st century, all, even the quantum, all of non-commutative geometry, all may be written in set theory, and all may be axiomatized, all is ultimately language. Worst of all, just this which has no meaning when it is formalized: thus, shouldn't we argue very strongly to convince anyone about the nullity of the logic of the [?]... But in a corner of the old woods the unavoidable sophism: \enquote{the consistency of the system is a mathematical statement which necessarily has a meaning}.

However, we have the impression that, as Thom said, the limits of truth, they are not falsity, they are insignificant\footnote{}. We could reasonably maintain that Gödel's theorem, before \enquote{I am not provable}, is first \enquote{I don't want to say anything}. This idea that anything () is \emph{a priori} true or false is a \enquote{fundamental intuition} which we can hardly get rid of. An analogy in the domain of moral: we have just experienced a Manichaean war, the axis of good against the axis of evil ---and

\section{Did God make the integers?}

\section*{Footnotes}

\end{document}
