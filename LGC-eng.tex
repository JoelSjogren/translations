\documentclass{article}
\usepackage{csquotes}

\author{Jean-Yves Girard}
\title{Logic as a geometry of cognition \\ (La Logique comme Géométrie du Cognitif)}
\date{22 March 2004}

\begin{document}

\maketitle
\noindent
(Unofficial translation.)

\section{The big cemeteries under the moon}
Replace the philosophy at the center of the scientific activity, restore the philosophy of science, as programme! For this, we propose to reactivate the major tool which constitutes \emph{logic}, by freeing it from the rut of the \enquote{linguistic turn}; this reactivation would be by means of geometry, a \enquote{geometric turn} of a sort. The geometrization becomes possible---and above all necessary---through the eruption of the \emph{cognitive}, which makes obsolete the old paradigm of verismo/dualism/atomism/essentialism inherited from the 20th century, the century of \emph{scientism}.

Scientism seeks answers, while \emph{science} would instead seek questions. Around 1900, one would seek the \enqoute{ultimate solutions}, all questions must find their answers, a little like maidens their husbands. Scientism, which cannot make any mistakes, tackled racial issues in 1904: 80,000 Hereros hanging in Namibia, in the name of eugenics... and it was nothing like the third Reich. And what to say about the final solution to class inequality, the extermination of the \enquote{Koulaks}, in the name of Marxist \enquote{science}, in the early 1930s? The paradigm under our critique is not affiliated with any of the murderous \enquote{-isms} of the past century; but it is fed by the same breast, an unprecedented regression of thought, accompanied by technical progress, also unprecedented.

\section{An equally long absence}
\section{The linguistic turn}
\section{Augustine vs Thomas}
\section{The cognitive}
\section{The (anti-)cognitive verismo}
\section{The information challenge}
\section{Set-theoretic atomism}
\section{Non-commutative geometry}
\section{The geometric turn}
\section{Object vs subject}
\section{The fundamental intuitions}
\section{Did God make the integers?}
\section*{Footnotes}

\end{document}
