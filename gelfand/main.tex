\documentclass{article}
\usepackage[T1]{fontenc}
\usepackage[style=french]{csquotes}
\usepackage{amsmath}
\usepackage{amssymb}
\usepackage{amsthm}

\usepackage{hyperref}


\newtheorem{theorem}{Theorem}
\newtheorem{corollary}{Corollary}
\newtheorem{lemma}{Lemma}

\theoremstyle{definition}
\newtheorem*{definition}{Definition}


\newcommand{\intd}[3]{\int_{#1}{#2}\mathrm{d}#3}

%\renewcommand{\thefootnote}{ \arabic{footnote})}

\title{Normed rings\footnote{A summary of this work (without the last two sections) was published in the \href{https://books.google.se/books/about/Comptes_rendus_de_l_Acad\%C3\%A9mie_des_scienc.html?id=wIkIAAAAIAAJ}{C. R. de l' Ac. d. Sc. de l' U. R. S. S., XXIII, N. 5, (1939)}.}}
\author{I. Gelfand (Moscow)}
\date{}%1941}

\begin{document}
\maketitle

\section{The axiomatics}
A totality $R$ of elements $x$, $y$, \dots is called a \emph{normed ring}, when:
\begin{enumerate}
\item [($\alpha$)] $R$ is a linear, normed, complete space in the sense of Banach.
\item [($\beta$)] A multiplication operation is defined on the elements of $R$, satisfying the usual algebraic properties
  \begin{align*}
    x(\lambda y + \mu z) &= \lambda xy + \mu xz,\\
    x(yz) &= (xy)z.
  \end{align*}
\item [($\gamma$)] There is a unit in $R$, that is, an element $e\in R$ such that $ex=x$, and thus $\| x \| \neq 0$.
\item [($\delta$)]
  The multiplication operation is continuous with respect to each factor: if $x_n \to x$ then $x_n y \to xy$, and if $y_n \to y$ then $xy_n \to xy$.
  
  In what follows the multiplication operation will be seen as commutative; however the results of section 1 don't depend on commutativity.

  In what follows we will use axiom ($\delta$) in the following form:
\item [($\delta'$)] $\|x\cdot y\| \leq \|x\|\cdot\|y\|$ and $\|e\| = 1$.
\end{enumerate}
A priori it would seem that ($\delta'$) is an essential strengthening of ($\delta$); we will show, however, that ($\delta'$) and ($\delta$) are in fact equivalent. Namely we have
\begin{theorem}
  For each normed ring $R$ one can find an isomorphic and homeomorphic ring $R'$, in which the axiom $(\delta')$ holds true.
\end{theorem}
Let $Q$ be the ring of linear operators on $R$ (seen as a linear space). To each element $x$ we can assign the linear operator $A_x\in Q$, defined by
$$A_x y = x\cdot y.$$

Clearly, distinct elements are sent to distinct operators, sums and products are sent to sums and products, and the unit $e$ is sent to the identity operator $E$.

Thus there exists within $Q$ a subring $R'$ isomorphic to $R$. Let's find out which property characterizes the linear operators $A\in R'$.

If $A\in R'$, then $A=A_x$, and hence
$$ A(yz) = x\cdot yz = (xy)z = Ay\cdot z $$
for any $y$ and $z$ in $R$. Let's show that if a linear operator $A\in Q$ has the property $A(yz)=Ay\cdot z$ (for any $y$ and $z$ in $R$) then $A\in R'$. In fact, let $Ae=x$; then $Ay=A(ey)=Ae\cdot y=xy$ for all $y$.

From this criterion it follows easily, that $R'$ is a closed subset of $Q$. A sequence $A_n\in R'$ may converge to $A\in Q$. Then $$A(x\cdot y)=\lim A_n(xy)=\lim (A_nx)y=Ax\cdot y,$$ from which, after what has been shown, $A\in R'$ follows. So $R'$ is a complete space. Suppose $\|e\|=a$. Then for $A_x\in R'$
$$ \|A_x\| = \sup_{\|y\|\leq1}\|A_xy\| = \sup_{\|y\|\leq1}\|x\cdot y\| \geq \frac1a \|x\cdot e\| = \frac1a\|x\|. $$
This means that the map from $R$ to $R'$ is continuous. Since the spaces $R$ and $R'$ are complete, it follows from the continuity of the map from $R$ to $R'$ that, by a well-known theorem of Banach's, the inverse mapping is continuous; thus we obtain that $R$ is homeomorphic with $R'$. But for elements $A\in R'$ as linear operators, axiom ($\delta'$) is obviously satisfied.
\begin{corollary}
  The product $xy$ is simultaneously continuous in both factors; if $x_n \to x$ and $y_n \to y$ then $x_ny_n \to xy$.
\end{corollary}
\begin{corollary}
  When dealing with absolutely convergent series one can, without changing the result, rearrange the terms at will; such series can be multiplied (added), and the series so obtained converges to the product (sum) of the convergent series.
\end{corollary}

\section{Preliminaries about ideals}
\begin{lemma}
  If $x\in R$ and $\|e - x\| < 1$ then $x$ has an inverse element.
\end{lemma}
To prove this we consider the series
$$ e + (e-x) + (e-x)^2 + \dots $$
Since $\|(e - x)^n\| < \|(e - x)\|^n$, this series converges; let $y$ be its sum. If we form the product of $y$ and $x=e-(e-x)$ and use corollary 2 of theorem 1, we get:
$$ yx = e + (e-x) + (e-x)^2 + \dots - (e-x) - (e-x)^2 - \dots = e. $$
Consequently $y = x^{-1}$.

The unit thereby possesses a neighbourhood $U(e)$, consisting of invertible elements. It is easy to realize that such a neighbourhood exists for every invertible $x$. Let $xy=e$ and let $U(x)$ be such a neighbourhood of $x$, that $U(x)y\subset U(e)$; if $z$ is any element of $U(x)$, then $z\cdot y$ has an inverse element $w$, $zyw=e$, from which it follows that the element $z$ also has an inverse. Consequently the set $V$ of invertible elements is open in $R$.

\begin{lemma}
  $x^{-1}$ is a continuous function of $x$ on $V$. In other words, if $x_n\in V$ converges, then $x_n^{-1} \to x^{-1}$.
\end{lemma}

Let $x=e$; then the sequence $x_n^{-1}$ is bounded, since for $\|x_n-e\|<\frac12$
$$ \|x_n^{-1}\| = \|e + (e-x_n) + \dots\| \leq 1 + \frac12 + \frac14 + \dots = 2. $$

Let $K=\sup\|x_n^{-1}\|$. Then
$$ \|e-x_n^{-1}\| = \|x_n^{-1}(x_n-e)\| \leq K\|x_n-e\| \to 0,$$
and consequently $x_n^{-1} \to e$.

In general the sequence $x_nx^{-1}$ converges to $xx^{-1}=e$, and therefore
$$ (x_n x^{-1})^{-1} = xx_n^{-1} \to e,\quad x_n^{-1}\to x^{-1}. $$

We called a totality $I$ of elements $x\in R$ an \emph{ideal} (or a \emph{non-trivial ideal}), when
\begin{enumerate}
\item [$1^\circ$.] From $x\in I$ and $y\in I$ it follows $px+qy\in I$, where $p$ and $q$ are arbitrary elements of the ring.
\item [$2^\circ$.] $I\neq R$.
\end{enumerate}

Clearly $I$ can't contain any elements of $V$, since otherwise it would contain any element $y\in R$:
$$ y=yx^{-1}\cdot x, $$
contradicting condition $2^\circ$. In every ring the element $0$ clearly forms an ideal; we will call it the zero ideal and denote it by $(0)$.

\begin{lemma}
  If a ring $R$ has no ideals, other than the zero ideal, then every $x\neq 0$ has an inverse $x^{-1}$; in other words, $R$ is in this case a field.
\end{lemma}

Let's consider the totality $I$ of elements $yx$, where $x\neq 0$, and $y$ runs through the whole ring $R$. Since $I\neq0$ (as $x\in I$), so $I = R$. This means, in particular, that some $y\in R$ exists, such that $xy=e$. Thereby $x$ possesses an inverse element.

We always have $I\subset R-V$; since $R-V$ is closed, we also have $\bar{I}\subset R-V$. Clearly $\bar{I}$ is also an ideal. We thus obtain
\begin{theorem}
  The closure $\bar{I}$ of a non-trivial ideal is again a non-trivial ideal.
\end{theorem}

Let an ideal $I$ be given. An element $x\in R$ is called congruent to $y\in R$ by the ideal $I$, $x\sim y$, when $x-y\in I$; as the notion of congruence is symmetric, reflexive and transitive, so $R$ is partitioned into classes of elements congruent by the ideal $I$; if we introduce in a natural way the operations of addition, subtraction and multiplication to these classes, then we obtain the \emph{residue class ring} $R/I$. This class has a unit, whose role is played by the class containing $e\in R$. The zero of this ring is that class, which consists of all $x\in I$.

\begin{lemma}
  If $I$ is a closed ideal, then $R/I$ is a normed ring.
\end{lemma}

We denote the elements of $R/I$ by $X$, $Y$, \dots and set $\|X\| = \inf_{x\in X} \|x\|$. Let's show that all axioms of the norm are satisfied:
\begin{enumerate}
\item [$1^\circ$.] $\|\lambda X\| = |\lambda| \|X\|$---is obvious.
\item [$2^\circ$.] $\|X+Y\| = \inf_{z\in X+Y} \|z\| = \inf_{x\in X, y\in Y} \|x + y\| \leq \inf_{x\in X, y\in Y} \{\|x\| + \|y\|\} \leq \inf_{x\in X} \|x\| + \inf_{y\in Y} \|y\| = \|X\| + \|Y\|$.
\item [$3^\circ$.] Similarly $\|X\cdot Y\| \leq \|X\|\cdot\|Y\|$.
\item [$4^\circ$.] Let $\|X\| = 0$, i.e. there exist a sequence $x_n\in X$ such that $x_n \to 0$. If $x$ is any element of $X$, then $x-x_n=y_n\in I$; since $x=\lim y_n$, so $x\in \bar{I}=I$. From here it follows, that the entire class $X$ is contained in $I$ and therefore coincides with it, i.e. is the zero of the ring $R/I$.
\item [$5^\circ$.] Let $E$ be the unit of $R/I$. Since $e\in E$, so $\|E\|\leq 1$; if it were the case that $\|E\|<1$, then there would exist an element of $E$, such that $\|x\|<1$. Then, according to lemma 1, the element $e-x$ would have an inverse, but this is impossible since $e-x\in I$.
\item [$6^\circ$.] \emph{Completeness}. Let $\|X_n-X_m\| \to 0$; choose a subsequence $X'_n$, such that the series $\sum\|X'_n-X'_{n+1}\|$ converges. For any $x_1\in X'_1$ one can find some $x_2\in X'_2$ such that
  $$ \|x_2-x_1\| < 2\|X'_2-X'_1\|, $$
  some $x_3\in X'_3$---such that
  $$ \|x_3-x_2\| < 2\|X'_3-X'_2\|, $$
  etc. The points $x_n$ form a Cauchy sequence [de: Fundamentalfolge]; thus
  $$ x_n \to x,\; X'_n\to X\ni x. $$
  Consequently also $X_n \to X$.
\end{enumerate}

\section{Normed fields}
Let $R$ be a complete normed space, and let an element $x\in R$ be given for each complex number $\lambda$ in a domain $\mathfrak{G}$ within the complex plane. The so-defined abstract function $x(\lambda)$ is called analytic, when for each $\lambda\in\mathfrak{G}$ the limit
$$ \lim_{h\to0}\frac{x(\lambda+h)-x(\lambda)}h $$
exists in the sense of convergence in norm.

Let $f$ be any linear functional on $R$; then $f(x(\lambda))$ is an ordinary analytic function of $\lambda$. This circumstance allows us to carry over many properties of ordinary analytic functions to abstract analytic functions.

\subsection{Liouville's theorem}
When $x(\lambda)$ is defined for all $\lambda$ and bounded in norm, then $x(\lambda)=x$, where $x$ is a constant element of $R$. In fact, for any $f$ we have $f(x(\lambda)=\textrm{const}$ by the ordinary Louville's theorem, and therefore $x(\lambda)$ is also constant, for if $x(\lambda)$ were to take two distinct values $x(\lambda_1)=x_1$ and $x(\lambda_2)=x_2$, then there would exist a linear functional by a theorem of Hahn's, such that $f(x_1)\neq f(x_2)$, which is impossible.

\subsection{Integrals}
Let a rectifiable curve $\Gamma$ be given inside the domain $\mathfrak{G}$. Then exists
$$ y = \intd{\Gamma}{x(\lambda)}{\lambda}. $$

Let $\lambda_0, \lambda_1, \dots, \lambda_n$ be arbitrary points on $Gamma$. Consider the sum
$$ S = \sum_{k=1}^n x(\lambda_k)(\lambda_k-\lambda_{k+1}). $$
Now let $\lambda_{00}, \dots, \lambda_{10}, \dots, \dots, \lambda_{n-1,p_{n-1}},\lambda_n$ be another collection of points on $\Gamma$, containing the first collection, such that $\lambda_{00} = \lambda_0$, $\lambda_{10} = \lambda_1$, \dots; let's estimate the difference between the corresponding sums. It is equal to
$$ \|\sum (x(\lambda_k) - x(\lambda_{k,d}))(\lambda_{k,d} - \lambda_{k,d})\| \leq \omega(x)\Gamma, $$
where $\omega(x)$ is the greatest difference $\|x(\lambda_k) - x(\lambda_{k,d})\|$ and $Gamma$ is the length of the curve. Let
$$ \delta = \sup_{\lambda\in\Gamma} \min \rho(\lambda, \lambda_k). $$
We claim that the sums $S$ possess a limit as $\delta \to 0$. Let, indeed, any $\epsilon$ be given in advance. Let's find a $\delta>0$, such that from $\lambda'-\lambda''<\frac\delta2$ the inequality $\|x(\lambda'-\lambda'')\|<\epsilon$ would follow, and let's estimate the deviation among all of the sums with a given $\frac\delta2$. Let the sequences $\mu_k$ and $\lambda_k$ correspond to the sums $S_2$ and $S_1$. If the sequence $\mu_k$ contains the sequence $\lambda_k$, then $\|S_2-S_1\| < \epsilon\Gamma$. Otherwise we form the sequence $\nu_1, \nu_2, \dots, \nu_k, \dots$---the common refinement of the sequences $\lambda_k$ and $\mu_k$. For the corresponding sums we have
$$ \|S_2-S_1\| < \epsilon\Gamma, \quad \|S_3-S_2\| < \epsilon\Gamma, $$
from which
$$ \|S_3-S_1\| < \epsilon\Gamma $$
follows. Therefore the sums $S$ possess a limit as $\delta\to0$, which we call the integral.

\subsection{Cauchy's theorem}
If the function $x(\lambda)$ is analytic inside and on a closed curve $\Gamma$, then
$$ y = \intd{\Gamma}{x(\lambda)}{\lambda}. $$
In fact, for any linear functional $f$ we have, by the usual theorem of Cauchy's, $f(y)=0$; from here it follows that $y=0$.

Cauchy's integral formula
$$ x(\lambda) = \frac1{2\pi i}\intd{\Gamma}{\frac{x(\xi)}{\xi-\lambda}} $$
is shown similarly.

As a consequence of Cauchy's integral formula we obtain the existence of all derivatives of the function $x(\lambda)$ and the Taylor series expansion:
$$ x(\lambda) = x(\lambda_0) + x'(\lambda_0)(\lambda-\lambda_0) + \dots, $$
convergent in every circle where the function $x(\lambda)$ is analytic.

We now show the following
\begin{theorem}
  A normed field is isomorphic to the field $T$ of all complex numbers\footnote{This theorem was first proved by Mazur. His proof is different from ours.}.
\end{theorem}

In other words, every element $x$ can be obtained the form $x=\lambda e$, where $\lambda$ is a complex number.

Assume the opposite: for some $x$ and all $\lambda$ the difference $x-\lambda e$ is nonzero. Then $(x-\lambda e)^{-1}$ exists for all $\lambda$. Let's show that this is an analytic function of $\lambda$.

The expression
$$ \frac1h[(x-(\lambda+h)e)^{-1}-(x-\lambda e)^{-1}] = -[(x-\lambda e-he)^{-1}](x-\lambda e)^{-1} $$
has, according to lemma 2, a limit, namely $-(x-\lambda e)^{-2}$. For large $\lambda$ we have
$$ |(x-\lambda e)^{-1}|\cdot|\lambda^{-1}|\left|\left(\frac{x}\lambda-e\right)^{-1}\right| \to 0, $$
again by lemma 2; since $(x-\lambda e)^{-1}$ is bounded due to continuity, $(x-\lambda e)^{-1}$ is also bounded for all $\lambda$. Using Liouville's theorem, we find: $(x-\lambda e)^{-1}$ is constant and therefore equal to zero, since it has zero for a limit when $\lambda\to\infty$; but then also $e=(x-\lambda e)(x-\lambda e)^{-1}=0$, which is impossible.

\section{Maximal ideals}
An ideal $M\subset R$ is called \emph{maximal}, if it is not part of any other ideal.

\begin{theorem}
  Every maximal ideal $M$ is closed.
\end{theorem}

Otherwise $M$ would be, by theorem 2, part of the ideal $\bar{M}$, and then it couldn't be maximal.

\begin{theorem}
  Every ideal $I$ is contained in a maximal ideal.
\end{theorem}

This theorem is easily shown by transfinite induction. Let $x_1, x_2, \dots, x_\alpha, \dots = \{x_\alpha\}$ be a totally ordered sequence of all elements of $R$. Each nonmaximal ideal $A$ can be assigned an ideal $A^+\supset A$ in the following way: consider the set of all elements $x\in\{x_\alpha\}$ such that the totality of elements $a+px\;(a\in A, p\in R)$ forms an ideal $\supset A$; by assumption this set is non-empty and possesses therefore a first element $x_A$. Set $A^+ = \{a+px_A\}$ and construct a transfinite sequence of ideals $I_\alpha$ in the following way: we set $I_0=I$; let $I_\alpha$ have been constructed for all $\alpha < \beta$; if $\beta$ belongs to the first class, that is, if some $\alpha=\beta-1$ exists, then we set $I_\beta = I_\alpha^+$; if $\beta$ belongs to the second class, then we set $$I_\beta = \sum_{\alpha < \beta} I_\alpha.$$ This sequence has cardinality which does not exceed the cardinality of $R$; therefore the sequence must have a last entry, and this is the maximal ideal that we were looking for.

For various applications the following simple theorem is important:
\begin{theorem}
  In order for an element $x\in R$ to have an inverse, it is necessary and sufficient that $x$ not belong to any ideal.
\end{theorem}

If $x$ has an inverse, then $x$ doesn't belong to any ideal; so it doesn't belong to any maximal ideal either.

If $x$ has no inverse, then the totality of elements $\{x\cdot y\}$, where $y$ runs through all elements of $R$, forms an ideal; according to theorem $5$ it is contained in a maximal ideal.

\section{Functions on maximal ideals}
Let $M$ be a maximal ideal of the ring $R$, and $R/M$---the residue class ring.
\begin{theorem}
  $R/M$ is isomorphic to the field of complex numbers.
\end{theorem}

By theorems 3 and 4 and lemma 3 it suffices to show, that $R/M$ contains no ideals other than the zero ideal. This much is clear however: if there were an ideal $I$ in $R/M$, that were separate from the zero ideal, then we would obtain, by union of all residue classes making up I, an ideal of $R$ containing $M$, which contradicts the maximality of the latter.

The converse claim is also true: If the residue class ring $R/M$ is isomorphic to the field of complex numbers, then $M$ is a maximal ideal. For otherwise, we would consider some ideal $M'\supset M$, and form the union of all classes in $R/M$ containing $M'$; thereby we would obtain an ideal in the field of complex numbers, which is impossible.

Theorem 7 allows us to assign to each element $x\in R$ a complex number $x(M)$, namely the one that represents the residue class containing $x$ in the isomorphism $R/M\approx T$.

For fixed $x$ and varying $M$ we obtain a function $x(M)$, which is defined on the set $\mathfrak{M}$ of all maximal ideals of the ring $R$.

These functions have the following properties:

\begin{enumerate}
\item [($\alpha$)] If $x=x_1+x_2$ then $x(M)=x_1(M)+x_2(M)$.
\item [($\beta$)] If $x=x_1\cdot x_2$ then $x(M)=x_1(M)\cdot x_2(M)$.
\item [($\gamma$)] $e(M)=1$.
\item [($\delta$)] $|x(M)| \leq \|x\|$.
\item [($\epsilon$)] If $M_1\neq M_2$ then there is an $x\in R$ such that $x(M_1)\neq x(M_2)$.

  These properties are just simple formulations of the fact that the mapping $R\to R/M$ is a homomorphism.
\item [($\zeta$)] If $x(M)$ does not vanish, then there exists some $y\in R$, such that $$ y(M) = \frac1{x(M)}.$$ This is a consequence of theorem 6.
\end{enumerate}

\section{The radical of a normed ring}
We commit to finding out, for which elements $x\in R$ the function $x(M)$ is 0; in other words, which elements belong to all maximal ideals.

\begin{definition}
  An element $x\in R$ is called a \emph{generalized nilpotent element}, if $\sqrt[n]{\|x^n\|} \to 0$. It is clear that the usual nilpotent elements (i.e. those for which $x^n = 0$ for some particular $n$) are also generalized nilpotent elements; the converse claim is in general not valid. The totality of generalized nilpotent elements is called the \emph{radical of the ring}.
\end{definition}

\begin{theorem}
  The intersection of all maximal ideals coincides with the set of all generalized nilpotent elements.
\end{theorem}

Let $x$ not belong to a certain maximal ideal $M_0$; then $x(M_0)\neq0$. Since
$$ \|x^n\| \geq |x^n(M_0)| = |x(M_0)|^n $$
we have
$$ \sqrt[n]{\|x^n\|} \geq |x(M_0)|; $$
thus $x$ is not a generalized nilpotent element. Let now $x$ belong to all maximal ideals. Then $e-\lambda x$ belongs to no maximal ideal for any $\lambda$, and it hence possesses an inverse element; the analytic function
$$ (e-\lambda x)^{-1} $$
is therefore entire and, by section 3, can be developed into a Taylor series:
$$ (e-\lambda x)^{-1}  = e + \lambda x + \lambda^2 x^2 + \dots $$
Since $\lambda^n x^n$ is a term in a convergent series, we have for $n\to\infty$
$$ \||\lambda^n x^n\|| \to 0. $$
From this it follows that
$$ \|x^n\| = \frac{\|\lambda^nx^n\|}{|\lambda^n|} < \frac1{|\lambda|^n} $$
for large enough $n$, and thus
$$ \lim_{n\to\infty}\sqrt[n]{\|x^n\|} \leq \frac1{|\lambda|}; $$
but since $\lambda$ was arbitrary, we obtain
$$ \lim_{n\to\infty}\sqrt[n]{\|x^n\|} = 0. $$

A more general theorem also holds, one which connects the maximum $|x(M)|$ to the norms of powers of $x$:
\newtheorem*{theorem8'}{Theorem 8'}
\begin{theorem8'}
  For each $x\in R$ the limit $$\lim\sqrt[n]{\|x^n\|}$$ exists and equals $|x(M)|$.
\end{theorem8'}

Let's put $\max |x(M)| = a$. Then the element $x-\mu e$ has an inverse for all $\mu$ with $|\mu|>a$, as it is represented by the function $x(M)-\mu$, which is everywhere non-zero. Let's put $\lambda=\frac1\mu$; then $(e\mu-x)^-1=\lambda(e-\lambda x)^{-1}$ is an analytic function on the disk $|\lambda|<\frac1a$. Develop it into a Taylor series:
$$ \lambda(e-\lambda x)^{-1} = \lambda(e + \lambda x + ...). $$
Since the general term of a convergent series tends to zero, so we obtain:
$$ \|\lambda^n x^n\| \to 0,\quad \|x^n\| = \frac1{|\lambda|^n}\|\lambda^n x^n\| \leq \frac1{|\lambda|^n} $$
for large enough $n$; from this it follows that
$$ \lim\sqrt[n]{\|x^n\|} \leq \frac1{|\lambda|}. $$
For $\lambda \to \frac1a$ we obtain
\[ \lim\sqrt[n]{\|x^n\|} \leq a. \tag{1} \]
As $\|x\|\geq a$, $\|x^n\|\geq a^n$ we have $\sqrt[n]{\|x^n\|}\geq a$ for all $n$, from which
\[ \lim\sqrt[n]{\|x^n\|} \geq a \tag{2} \]
follows. Comparing (1) and (2) we find:
$$ \lim\sqrt[n]{\|x^n\|} = a = \max |x(M)|. $$

\section{Topologization of the set $\mathfrak{M}$}
As we intend to make the functions $x(M)$ into continuous functions, we introduce a topology on the set $\mathfrak{M}$.

We define a neighbourhood $U(M_0)$ of a point $M_0\in\mathfrak{M}$ to be a totality of all $M\in\mathfrak{M}$ for which the inequalities
$$ |x_i(M) - x_i(M')| < \epsilon \quad (i=1,2,\dots,n)$$
hold. In this way the neighbourhood is defined by providing $\epsilon$ and any elements of $R$.

Let's show, that the axioms of topological space are satisfied.
\begin{enumerate}
\item [$1^\circ$.] Each point has a neighbourhood and is contained in it.
\item [$2^\circ$.] The intersection of two neighbourhoods is another neighbourhood.
\item [$3^\circ$.] When $M_1\subset U(M_0)$, there exists $U(M_1)\subset U(M_0)$.
\end{enumerate}

It is clear, that the first two axioms are satisfied. In order to show that the third axiom is also satisfied, we find an $\epsilon'$ such that the circle of radius $\epsilon'$ centered at $x_i(M_1)$ is contained within the circle of radius $\epsilon$ centered at $x_i(M_0)$; it is then clear that
$$ U(M_1) = \{|x_i(M) - x_i(M_1)| < \epsilon'\} $$
satisfies the required condition.

\begin{theorem}
  The topological space $\mathfrak{M}$ is bicompact and enjoys the axioms given by Hausdorff.
\end{theorem}

The proof that we give here is based on Tychonoff's theorem on the bicompactness of topological products of bicompact spaces\footnote{\href{https://sci-hub.st/https://link.springer.com/article/10.1007/BF01782364}{\emph{A. Tychonoff}, Math. Annalen, 102, (1929)}.}. A proof that does not use Tychonoff's theorem can be found in \footnote{See \href{http://www.mathnet.ru/links/253be47c30469d06812d656be551fd8a/sm6047.pdf}{\emph{I. Gelfand} and \emph{G. \v{S}ilov}, On several methods of introducing a topology on the set of maximal ideals of a normed ring (in this volume of Recueil mathématique, p. 31)}.}.

Let's associate to each element $x\in R$ a disk $Q_x$ of radius $\|x\|$ in the complex plane, and consider the topological product $\Omega$ of all these disks. According to Tychonoff's theorem $\Omega$ is bicompact. To each $M_0\in\mathfrak{M}$ we associate that point in $\Omega$ which is determined by the values $x(M_0)$. From ($\epsilon$), section 5, follows that $\mathfrak{M}$ is mapped one-to-one into part of $\Omega$; the topology in $\mathfrak{M}$ coincides with the one induced on the image of $\mathfrak{M}$ in $\Omega$; thus $\mathfrak{M}$ is homeomorphic to part of $\Omega$. To show the bicompactness of $\mathfrak{M}$, it now suffices to show that it is closed in the bicompact space $\Omega$.

Let $\Lambda=\{\lambda_x\}$ be a limit point of the set $\mathfrak{M}$ in $\Omega$; we construct a maximal ideal $M_0\in\mathfrak{M}$ such that $x(M_0)=\lambda_x$; it will now be shown that $\Lambda\in\mathfrak{M}$. Let's show that $\lambda_{x+y}=\lambda_x+\lambda_y$. To this end we consider a neighbourhood of the point $\lambda$, that is determined by the points $x\cdot y$ and an arbitrarily small number $\epsilon$. Since $\lambda$ is a limit point of $\mathfrak{M}$, a point $M\in\mathfrak{M}$ will appear in its neighbourhood, such that
\begin{align*}
  |\lambda_x &- x(M)| < \epsilon \\
  |\lambda_y &- y(M)| < \epsilon \\
  |\lambda_{x+y} &- x(M) - y(M)| < \epsilon.
\end{align*}
Hence $|\lambda_{x+y} - \lambda_x - \lambda_y| < 3\epsilon$, and since $\epsilon$ was arbitrary, $\lambda_{x+y} = \lambda_x + \lambda_y$. We obtain, that the correspondence $x\to\lambda_x$ is a homomorphism from the ring $R$ to the field of complex numbers. In the same manner we show that $\lambda_{ie} = i$. Consequently there exists a maximal ideal $M_0\in\mathfrak{M}$ such that $$x(M_0)=\lambda_x$$ for all $x$, and the theorem has been proven.

Furthermore let's consider, that all functions $x(M)$ are automatically continuous: in order to find a neighbourhood $V(M_0)$, in which the function $x(M)$ does not vary more than $\epsilon$, it suffices to consider the set
$$ V(M_0) = \{|x(M) - x(M_0)| < \epsilon\}, $$
which is a neighbourhood by definition.

Thus we obtain
\begin{theorem}
  Every normed ring $R$ can be mapped homomorphically into the ring $C(\mathfrak{M})$ of continuous functions defined on the bicompact Hausdorff space $\mathfrak{M}$. In order for the map from $R$ to $C(\mathfrak{M})$ to be an isomorphism, it is necessary and sufficient that the following condition is satisfied: from $\sqrt[n]{\|x^n\|} \to 0$ follows $x=0$.
\end{theorem}

The following theorem shows the necessity of this topology.

\newtheorem*{theorem10'}{Theorem 10'}
\begin{theorem10'}
  Let $\mathfrak{M}$ be topologized in any manner such that
  \begin{enumerate}
  \item [($\alpha$)] $\mathfrak{M}$ is bicompact,
  \item [($\beta$)] the functions $x(M)$ are continuous.
  \end{enumerate}
  Denote the topological space with the new topology by $\mathfrak{M_1}$ --- and with the old one --- by $\mathfrak{M_0}$. Then $\mathfrak{M_0}$ is isomorphic to $\mathfrak{M_1}$.
\end{theorem10'}
The map from $\mathfrak{M_0}$ into $\mathfrak{M_1}$, sending each $M\in\mathfrak{M_0}$ to the same in $\mathfrak{M_1}$, is one-to-one. From ($\beta$) the image in $\mathfrak{M_1}$ of any open set in $\mathfrak{M_0}$ is still open; consequently the map $\mathfrak{M_1}\to\mathfrak{M_0}$ is continuous. By a well-known theorem of topology, ($\alpha$) implies that the map is continuous in both directions, that is, $\mathfrak{M_0}$ is isomorphic to $\mathfrak{M_1}$.

Since we have applications of this theory in mind, we remark that it is not necessary to use all elements of the ring in the specification of the topology: one can limit oneself to the generating elements. A totality $K\subset R$ is called \emph{a totality of generators} of $R$, when the smallest closed subring containing $K$ coincides with $R$.

\begin{theorem}
  The totality of neighbourhoods
  $$ U(M_0) = \{|x_i(M) - x_i(M_0)| < \epsilon\},\quad i=1,2,\dots,n, $$
  where $\epsilon$ and $n$ are arbitrary, and $x_1, x_2, \dots, x_n$---are elements of $K$, is a defining system of neighbourhoods in $\mathfrak{M}$.
\end{theorem}

We must show, that in each neighbourhood
$$ U(M_1) = \{|x_i(M) - x_i(M_0)| < \epsilon\},\quad i=1,2,\dots,n, $$
$$ x_1, x_2, \dots, x_n\in R, $$
there is a neighbourhood from the defining system.

Let's find $n$ polynomials
$$ P_1(x_{11}, x_{12}, \dots, x_{1n_1}), \dots, P_n(x_{n1}, x_{n2}, \dots, x_{nn_n}) $$
in the elements $x_ik\in K$, that are less than $\frac\epsilon3$ in norm in their separation from the elements $x_1, x_2, \dots, x_n$. For each polynomial we find such a $\delta$, that the deviation of the arguments $x_{k1}(M), \dots, x_{kn_k}(M)$ from the values $x_{k1}(M_0), \dots, x_{kn_k}(M_0)$ by less than $\delta$ does not cause the value of the polynomial to change by more than $\frac\epsilon3$.

[...]


\end{document}

\begin{theorem}
\end{theorem}

\begin{enumerate}
\item [($\alpha$)]
\end{enumerate}
