\documentclass{article}
\usepackage[T1]{fontenc}
\usepackage[style=french]{csquotes}
\usepackage{amsmath}
\usepackage{amssymb}
\usepackage{amsthm}

\newtheorem{theorem}{Theorem}

\theoremstyle{definition}
\newtheorem{definition}{Definition}

\newtheorem{innermanualdefinition}{Definition}
\newenvironment{manualdefinition}[1]
  {\renewcommand\theinnermanualdefinition{#1}\innermanualdefinition}
  {\endinnermanualdefinition}

\newtheorem{innermanualtheorem}{Theorem}
\newenvironment{manualtheorem}[1]
  {\renewcommand\theinnermanualtheorem{#1}\innermanualtheorem}
  {\endinnermanualtheorem}

\renewcommand{\thefootnote}{ \arabic{footnote})}

\author{\\[-2em]{\small by} \\[0.5em] Alfred Tarski (Warsaw).}
\title{Basics of the system calculus. \\ (Grundzüge des Systemenkalküls.) \\ {\small Part 1\footnote{Note from the editor: The second part of this treatment arrived at the same time as the first one and will --- in agreement with the author --- be published in the next volume of ``Fundamenta Mathematicae''.}\;).}}
\date{}%1935}

\begin{document}

\maketitle

\noindent
[...]

\section{Basic concepts and axioms}

[...]

\section{The system calculus}

[...]

\newcommand{\no}[1]{\;\textsuperscript{#1})\;}
\newcommand{\tss}[1]{\textsuperscript{#1}}

\vspace{1em}
\textit{Postulate VII.\, If $x\in B$, then \no{a} $\overline{x}\in B$, \no{b} $x \cdot \overline{x}=0$, and \no{c} $x + \overline{x}=1$.}

\vspace{0.5em}
The system of the algebra of logic can be extended, by introducing two infinite operations: ``$\sum_{y\in X} y$'' --- ``the sum of all things in the set $X$'' and ``$\prod_{y\in X} y$'' --- ``the product of all things in the set $X$''. It is then necessary to add the following postulates:

\vspace{0.5em}
\textit{Postulate VIII.\, If $X\subset B$, then \no{a} $\sum_{y\in X} y\in B$ and \no{b} $x < \sum_{y\in X} y$ for all $x\in X$; \no{c} if furthermore $z\in B$ and $x<z$ for all $x\in X$, then $\sum_{y\in X} y<z$.}

\textit{Postulate IX.\, If $X\subset B$, then \no{a} $\prod_{y\in X} y\in B$ and \no{b} $\sum_{y\in X} y < x$ for all $x\in X$; \no{c} if furthermore $z\in B$ and $z<x$ for all $x\in X$, then $z < \prod_{y\in X} y$.}

\textit{Postulate X. If $x\in B$ and $X\subset B$, then \no{a} $x \cdot \sum_{y\in X} y = \sum_{y\in X}(x \cdot y)$ and\no{b} $x + \prod_{y\in X} y = \prod_{y\in X}(x + y)$.}

\vspace{0.5em}
The postulates I--VII, together with all the theorems that follow from them, form the \emph{common} and the postulates I--X the \emph{extended system of the algebra of logic}\footnote{Compare with $T_5$, p.\,177--180. One should beware that the postulates I--X are not mutually independent; in particular one can show that any three out of the four postulates V\tss{a}, V\tss{b}, X\tss{a} and X\tss{b} can be omitted from the rest of the postulates of the system considered (this sharpens a remark made in $T_5$, p.\,180).}.

We proceed to the calculus of propositions, which we will mention only briefly; in order to avoid misunderstandings, we will not simply say propositional calculus, but rather \emph{algorithm of propositions}: the jargon ``propositional calculus'' already has a definite meaning, which different from what we are intending.

The range of considerations of the algorithm of propositions is formed by the set $S$. We will define two relations between the elements of this set: ``$x\supset y$'' (``$x$ implies $y$'') and ``$x=y$'' (``$x$ is equivalent to $y$'').

\newcommand{\then}{\rightarrow}

\begin{manualdefinition}{4} \no{a} $x\supset y$ holds precisely when $x,y\in S$ and $x\then y\in L$; \no{b} $x=y$ holds precisely when both $x\supset y$ and $y\supset x$ hold.
\end{manualdefinition}

Then one can prove the following:

\begin{manualtheorem}{4} We replace the the symbols ``B'', ``<'' and ``='' by ``S'', ``$\supset$'' and ``\equiv$'' in all postulates of the common systems of the algebra of logic; furthermore we replace ``0'' by the variable ``$u$'' and ``1'' by the variable ``$v$'', and we make the assumptions $u\in S$ and $\overline{u}\in L$ as well as $v\in L$ everywhere in the postulates; finally we leave the remaining symbols unchanged. Then the postulates I--VII are satisfied.
\end{manualtheorem}


\vspace{0.5em}\noindent
[...]

\end{document}




\begin{theorem}
\end{theorem}
